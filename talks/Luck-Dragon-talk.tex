\documentclass{beamer}

\mode<presentation>
{
  \usetheme{Warsaw}
  \definecolor{links}{HTML}{2A1B81}
  \hypersetup{colorlinks,linkcolor=,urlcolor=links}
  %\usetheme{Frankfurt}
  \usecolortheme{seagull}
  % or ...
  \setbeamercovered{transparent}
  % or whatever (possibly just delete it)
}

\usepackage[english]{babel}
\usepackage[utf8]{inputenc}
\usepackage{times}
\usepackage[T1]{fontenc}
\usepackage{fancyvrb}
\usepackage{listings}
\usepackage{graphicx}
\usepackage{attachfile}
\usepackage{ifthen}

\newboolean{localPieces} %Declaration, defaults to false
\setboolean{localPieces}{false} %Assignment

\title[Luck Dragon] % (optional, use only with long paper titles)
{Luck Dragon Talk}

\author[Gogins] % (optional, use only with lots of authors)
{Michael Gogins \\ \url{http://michaelgogins.tumblr.com} }
% - Give the names in the same order as the appear in the paper.
% - Use the \inst{?} command only if the authors have different
%   affiliation.

\institute[Irreducible Productions] % (optional, but mostly needed)
{
  Irreducible Productions\\
  New York
}
% - Use the \inst command only if there are several affiliations.
% - Keep it simple, no one is interested in your street address.

\date[27 October 2023] % (optional, should be abbreviation of conference name)
{27 October 2023}
% - Either use conference name or its abbreviation.
% - Not really informative to the audience, more for people (including
%   yourself) who are reading the slides online

\subject{Computer Music}
\expandafter\def\expandafter\insertshorttitle\expandafter{%
    \insertshorttitle\hfill%
    \insertframenumber\,/\,\inserttotalframenumber}
% This is only inserted into the PDF information catalog. Can be left
% out. 
\begin{document}
\lstset{basicstyle=\ttfamily\tiny,commentstyle=\ttfamily\tiny,tabsize=2,breaklines,fontadjust=true,keepspaces=false,fancyvrb=true,showstringspaces=false,moredelim=[is][\textbf]{\\emph\{}{\}}}

\begin{frame}
  \titlepage
\end{frame}

\begin{frame}{Outline}
	I will talk about my way of doing algorithmic composition. Almost all my pieces use \href{http://csound.github.io}{Csound} for synthesis, but I use it with other languages for score generation, including Python, C++, Lua, Lisp, JavaScript, and recently \href{https://github.com/tidalcycles/strudel}{Strudel} in \href{https://gogins.github.io/cloud-5.html}{cloud-5}, which I will demonstrate. There are hyperlinks in these slides to external resources.
\tableofcontents
  % You might wish to add the option [pausesections]
\end{frame}

\section{About Me}
\begin{frame}{About Me}

\begin{itemize}
\item I was born in 1950 in Salt Lake City. I've lived in Minneapolis, Sonoma County, LA, Seattle, and NYC.
\item My wife Heidi and I own a farm in Bovina, and we keep a co-op on the Upper West Side.
%\item I have had too many interests, but I was always interested in the future and in music.
\item While getting my BA in comparative religion at the University of Washington, I studied computer music with John Rahn.
\item Computer music gradually but completely took over. I programmed trading systems as a day job.
\item \emph{Note well}: I was never either an academic or, after a few years in LA, a performing musician.
\item More about me \href{https://michaelgogins.tumblr.com}{here}.
\end{itemize}

\end{frame}

\section{cloud-5}
\begin{frame}{cloud-5}

\begin{itemize}
\item \href{https://gogins.github.io/cloud-5.html}{cloud-5} is my system/toolkit for making \emph{real computer music entirely in the HTML5 environment}.
\item That is, all cloud-5 pieces run (sometimes forever!) in an ordinary Web browser, and they have \emph{no} external dependencies.
\item Pieces are hosted either on a regular Web server on the Internet, or on a local Web server.
\item \emph{Only} limitation: Pieces can't read or write on the filesystem.
\item cloud-5 includes \href{https://github.com/gogins/csound-wasm}{WebAssembly builds} of \href{http://csound.github.io}{Csound} and my algorithmic composition system \href{https://github.com/gogins/csound-ac}{CsoundAC} that works with chord spaces, the live coding system \href{https://github.com/tidalcycles/strudel}{Strudel}, and some standard JavaScript libraries -- all packaged in the cloud-5 release.
\item I will now perform \href{http://gogins.github.io/cancycle.html}{\emph{Cancycle}}.
\end{itemize}

\end{frame}

\section{Getting Started}
\begin{frame}{Getting Started with cloud-5}

\begin{enumerate}
\item Download cloud-5.zip from \href{https://github.com/gogins/cloud-5/releases}{GitHub}, in the \emph{Assets} menu, bottom of page.
\item Unzip cloud-5.zip on your computer.
\item In a terminal, change to the \texttt{cloud-5/cloud-5} directory and run a Web server, e.g.: \texttt{python3 -m http.server}.
\item Open a Web browser to \url{http://localhost:8000} and play a sample piece.
\item If you write a new piece, adapt an existing piece using a text editor; save it in the  \texttt{cloud-5/cloud-5} directory.
\item Use the browser's developer tools to debug the piece.
\item You may need to clear the browser caches and do a hard refresh to see changes that you make.
\end{enumerate}

\end{frame}

\section{Tape Music}
\begin{frame}{Tape Music I}

\begin{itemize}
\item In the mists of time -- before computers, before even the Moog synthesizer -- there was \emph{electronic music}.
\item It was composed by splicing together snippets of recording tape. It could be musique concrète à la Pierre Schaeffer, or Stockhausen-type music made with oscillators and filters.
\item Even now that tape recorders are vanishing back into the mists, music made using technology to create a sound recording for playback is still sometimes called \emph{tape music}.
\item Most of my pieces are, in fact, tape music, \emph{because that gives the composer the greatest power}. In theory, no mistake goes unfixed, and no possible improvement is not found. Of course, this all takes absolutely forever....
\end{itemize}
\end{frame}

\begin{frame}{Tape Music II}

\begin{itemize}
\item I use all the same software for tape music pieces that I use in cloud-5... but I also:
\item Read and write soundfiles (and other files).
\item Use \href{https://www.zanderjaz.com/downloads/soundfonts}{sample banks}.
\item Use Csound plugins from \href{https://github.com/csound-plugins/risset-data/tree/master}{Risset} and \href{https://github.com/csound-plugins/csound-externals}{csound-externals}.
\item Use \href{https://github.com/gogins/csound-vst3-opcodes}{VST3 plugins}.
\item \href{https://github.com/gogins/csound-webserver-opcodes}{Embed HTML and JavaScript} in my pieces.
\item Even \href{https://github.com/gogins/csound-cxx-opcodes}{embed C++ code} in my pieces.
\end{itemize}
\end{frame}

\begin{frame}{Tape Music III}

\begin{itemize}
\item Some finished works: 
\begin{itemize}
\item \href{https://www.youtube.com/@michaelgogins}{On YouTube}, e.g.\href{https://music.youtube.com/watch?v=3_ahbL44p-E&si=2ScceuKnI0Pqye5G}{\emph{Two Dualities}}, operations in chord space. Source code \href{https://github.com/gogins/cloud-5/blob/main/talks/mkg-2009-09-14-o.py}{here}.
\item \href{https://music.amazon.com/artists/B0016KQMPA/michael-gogins}{On Amazon Music}, e.g. \href{https://music.amazon.com/albums/B0016UPQRK?trackAsin=B0016UGIW2\&do=play\&ref=dm_sh_26UFIwpvtSDIIQmF2rIxH1qXC}{\emph{csound-2005-03-06--03.38.19.py}}, \emph{In C} meets \emph{Musikalisches Würfelspiel}. Source code \href{https://github.com/gogins/cloud-5/blob/main/talks/csound-2005-03-06--03.38.19.py}{here}.

\end{itemize}
\end{itemize}

By the way, the source code for most of my software and many of my compositions is available in one or another of my repositories on \href{https://github.com/gogins}{GitHub}.


\end{frame}

%\section{Questions?}
\begin{frame}{Questions?}

\end{frame}


\begin{frame}[allowframebreaks]
  \frametitle<presentation>{Resources}
    
  \begin{thebibliography}{10}
    
  \beamertemplatebookbibitems
  % Start with overview books.

  \bibitem{GBlog} \href{http://michaelgogins.tumblr.com}{Michael Gogins, blog}.
  
  \bibitem{GGithub} \href{https://github.com/gogins/gogins.github.io}{Michael Gogins. ``Computer Music Resources.''}

  \bibitem{CQT2008} \href{http://www.sciencemag.org/content/320/5874/346.abstract}{Clifton Callender, Ian Quinn, and Dmitri Tymoczko. ``Generalized voice-leading spaces.'' \emph{Science}, 320:346–
348, 2008.}

  \bibitem{G1991} {Michael Gogins. ``How I Became Obsessed with Finding a Mandelbrot Set for Sounds,'' \textbf{\textit{News of Music}} \textbf{13}:129-139.}

  \bibitem{FS2005} \href{http://www.mtosmt.org/issues/mto.05.11.3/mto.05.11.3.fiore_satyendra.pdf}{T.M. Fiore and R. Satyendra. ``Generalized Contextual
Groups.'' \emph{Music Theory Online}, 11(3), 2005}.

  \bibitem{G2006}
    \href{https://www.dropbox.com/s/ztej71n2fbn4tq4/Lindenmayer_Systems_Based_on_Riemannian_Transformations.pdf}{Michael Gogins. ``Score generation in voice-leading
and chord spaces.'' In Georg Essl and Ichiro Fujinaga,
editors, \emph{Proceedings of the 2006 International Computer Music Conference}, San Francisco, California,
2006. International Computer Music Association.}

  \bibitem{T2006} \href{http://www.sciencemag.org/content/313/5783/72.abstract?ijkey=wzKBea3ktKdu2&keytype=ref&siteid=sci}{Dmitri Tymoczko. ``The Geometry of Musical Chords.'' \emph{Science}, 313:72–74, 2006.}

  \end{thebibliography}

\end{frame}

\end{document}


